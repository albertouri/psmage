\documentclass[letterpaper]{article}
\usepackage{aaai}
\usepackage{times}
\usepackage{helvet}
\usepackage{courier}
\frenchspacing
\pdfinfo{
/Title (PSMAGE: Procedural Starcraft MAp GEneration)
/Subject (AAAI Publications)
/Author (Alberto Uriarte, Jordan Santell)}
\setcounter{secnumdepth}{0}  
 \begin{document}
% The file aaai.sty is the style file for AAAI Press 
% proceedings, working notes, and technical reports.
%
\title{PSMAGE: Procedural Starcraft MAp Generation}
\author{Alberto Uriarte \and Jordan Santell\\
Drexel University\\
Philadelphia\\
}
\maketitle
\begin{abstract}
\begin{quote}
Human map designer lost a lot of time tunning their maps to make it well balanced. This paper presents an algorithm for generating balanced maps for the popular real-time strategy (RTS) game StarCraft. We use Voronoi diagrams to randomly generate polygons as starting point. Then different properties are assigned to the polygons with some fitness functions in order to make consistent and balanced maps.
\end{quote}
\end{abstract}

\section{Introduction}
This sections are tentavie

\section{Problem definition}

\section{Related work}
Santi says that we don't have to do this kind of section...

\section{Procedural map generation}
The algorithm presented in this paper can be outlined as the following sequence of seven steps:
\begin{enumerate}
	\item Generate a random seed point
	\item Compute Voronoi diagram
	\item Add borders
	\item Compute region elevation
	\item Mirror map
	\item Parse output into a .chk file
	\item MPQ archive
\end{enumerate}

\subsection{1. Generate a random seed point}


\subsection{2. Compute Voronoi diagram}


\subsection{3. Add borders}


\subsection{4. Compute region elevation}


\subsection{5. Mirror map}


\subsection{6. Parse output into a .chk file}


\subsection{7. MPQ archive}


\section{Implementation} % (fold)
\label{sec:implementation}

% section implementation (end)

\section{Results} % (fold)
\label{sec:results}

% section results (end)

\section{Future research} % (fold)
\label{sec:future_research}

% section future_research (end)

\section{Conclusion} % (fold)
\label{sec:conclusion}

% section conclusion (end)


%%%%%%%%%%
% References and End of Paper
% \bibliography{psmage-bib}
% \bibliographystyle{aaai}
\end{document}